\documentclass[compress]{beamer}

\usetheme{Hamburg}

\usepackage[T1]{fontenc}
\usepackage[utf8]{inputenc}

\usepackage{lmodern}

%\usepackage[english]{babel}
\usepackage[ngerman]{babel}

\usepackage{eurosym}
\usepackage{listings}
\usepackage{microtype}
\usepackage{units}

\renewcommand*{\thefootnote}{\fnsymbol{footnote}}

\lstset{
	basicstyle=\ttfamily\footnotesize,
	frame=single,
	numbers=left,
	language=C,
	breaklines=true,
	breakatwhitespace=true,
	postbreak=\hbox{$\hookrightarrow$ },
	showstringspaces=false,
	tabsize=4,
	captionpos=b,
	morekeywords={gboolean,gpointer,gconstpointer,gchar,guchar,gint,guint,gshort,gushort,glong,gulong,gint8,guint8,gint16,guint16,gint32,guint32,gint64,guint64,gfloat,gdouble,gsize,gssize,goffset,gintptr,guintptr,int8_t,uint8_t,int16_t,uint16_t,int32_t,uint32_t,int64_t,uint64_t,size_t,ssize_t,off_t,intptr_t,uintptr_t,mode_t}
}

\title{SCIL - Scientific Compression Interface Library}
\author{Armin Schaare}
\institute{Arbeitsbereich Wissenschaftliches Rechnen\\Fachbereich Informatik\\Fakultät für Mathematik, Informatik und Naturwissenschaften\\Universität Hamburg}
\date{2016-03-28}

\titlegraphic{\includegraphics[width=0.75\textwidth]{logo}}

\begin{document}

\begin{frame}
	\titlepage
\end{frame}

\begin{frame}
	\frametitle{Outline}

	\tableofcontents[hidesubsections]
\end{frame}

\section{Introduction}
\subsection*{}

\begin{frame}
	\frametitle{What is SCIL?}

	\begin{itemize}
		\item Lossy and lossless compression library
		\item Uses many compression algorithms and can be extended
		\item User provides parameters for compression
		\item Determines best suited compression algorithm automatically
	\end{itemize}

\end{frame}

\begin{frame}
	\frametitle{User provided compression parameters}

	\begin{itemize}
		\item Absolute/relative error tolerance
		\item Minimum performance/throughput
		\item Dimensional configuration of the data
		\item etc...
	\end{itemize}
\end{frame}

\section{Workflow \& Usage}
\subsection*{}

\begin{frame}
	\frametitle{Compression Algorithms}

	SCIL uses a repertoire of compression algorithms.\\
	Currently:

	\bigskip

	\begin{itemize}
		\item (memcopy)
		\item abstol \footnote{Own developement}
		\item gzip
		\item sigbits \footnotemark[\value{footnote}]
		\item fpzip
		\item zfp
	\end{itemize}

	\setcounter{footnote}{0}
\end{frame}

\begin{frame}
	\frametitle{Algorithm selection}

	Before using SCIL run ./scil-config

	\bigskip

	\begin{itemize}
		\item Generates performance data for each algorithm
		\item Platform dependent
		\item SCIL uses data to determine best algorithm each compression
	\end{itemize}

	\footnote{Not implemented yet}
\end{frame}

\begin{frame}
	\frametitle{Usage}

	\begin{itemize}
		\item Provide compression parameters
		\item Initialize compression context
		\item Provide data
		\item Compress data
	\end{itemize}
\end{frame}

\begin{frame}[fragile]
	\frametitle{Usage Example Code - 1}

	\begin{lstlisting}[caption=SCIL usage example - 1]
	scil_hints hints;
	hints.absolute_tolerance = 0.005;
	hints.relative_tolerance_percent = 1.0;
	hints.significant_bits = 5;

	scil_context * ctx;
	scilPr_create_context(&ctx, &hints);

	size_t* lengths = (size_t*)malloc(100 * sizeof(double));
	scil_dims_t_t dims = scil_init_dims(1, lengths);
	...
	\end{lstlisting}

\end{frame}

\begin{frame}[fragile]
	\frametitle{Usage Example Code - 2}

	\begin{lstlisting}[caption=SCIL usage example - 2]
	...

	double* buffer_in = (double*)malloc(100 * sizeof(double));
	byte* buffer_out = (byte*)malloc(100 * sizeof(double) + SCIL_BLOCK_HEADER_MAX_SIZE);

	size_t c_size;
	scil_compress(SCIL_DOUBLE, buffer_out, &c_size, buffer_in, dims, ctx);
	\end{lstlisting}

\end{frame}

\begin{frame}[fragile]
	\frametitle{Decompression}

	\begin{itemize}
		\item Uses relevant information for decompressing data
		\item Relevant information is stored in header of compressed buffer
		\begin{itemize}
			\item Which algorithm to use
			\item Absolute/Relative error tolerance
			\item Minimum value
			\item Dimensional configuration of data
			\item etc...
		\end{itemize}
		\item User only needs to provide compressed buffer and its size
	\end{itemize}

\end{frame}

\section{Conclusion}
\subsection*{Benchmarks}

\begin{frame}
	\frametitle{Compression Throughput}

	\begin{center}
		\includegraphics[width=\textwidth,height=0.8\textheight,keepaspectratio]{throughputs.png}
	\end{center}

\end{frame}

\begin{frame}
	\frametitle{Compression Ratios}

	\begin{center}
		\includegraphics[width=\textwidth,height=0.8\textheight,keepaspectratio]{compression_rates.png}
	\end{center}

\end{frame}

\begin{frame}
	\frametitle{Conclusion}

	Core concepts:
	\begin{itemize}
		\item Adaptive compression system
		\item Dynamic throughput and compression speeds
	\end{itemize}

	\bigskip

	Future development:
	\begin{itemize}
		\item Multi-dimensional data support
		\begin{itemize}
			\item 2D
			\item 3D
			\item Icosahedral
		\end{itemize}
		\item Automatic, optimal algorithm choice
	\end{itemize}
\end{frame}

\section{Literature}
\subsection*{}

\nocite{*}

\begin{frame}
	\frametitle{Literature}

	\bibliographystyle{alpha}
	\bibliography{literatur}
\end{frame}

\appendix

\begin{frame}
	\frametitle{Abstol - Needed Bit Count}

	\begin{center}
		$N_{bits}=\lceil\log_2{\left(1 + \frac{V_{max} - V_{min}}{2 * \Delta{e}}\right)}\rceil$
	\end{center}

	\bigskip

	, where\\

	\bigskip

	\begin{tabular}{ll}
		$V_{min}$ & Minimum value in data \\
		$V_{max}$ & Maximum value in data \\
		$\Delta{e}$ & Absolute error tolerance \\
		$N_{bits}$ & Needed number of bits per value
	\end{tabular}

\end{frame}

\begin{frame}
	\frametitle{Abstol - Encoding}

	\begin{center}
		$V_e = R\left(\frac{V_i - V_{min}}{2 * \Delta{e}}\right)$
	\end{center}

	\bigskip

	, where\\

	\bigskip

	\begin{tabular}{ll}
		$V_i$ & Current value \\
		$V_e$ & Encoded value \\
		$R\left(x\right)$ & Round to nearest integer function
	\end{tabular}

\end{frame}

\begin{frame}
	\frametitle{Sigbits}

	\begin{itemize}
		\item Calculate significant bits, from ...
		\begin{itemize}
			\item ... significant digits and
			\item ... relative error tolerance
		\end{itemize}
		\item Find minimum and maximum exponent and sign
		\item Calculate needed bit count for both
		\item Encode sign and exponent
		\item Strip mantissa of dispensable bits
		\item Pack buffer tightly (bitwise)
	\end{itemize}

\end{frame}

\begin{frame}
	\frametitle{Sigbits - Significant Bits Formula}

	From significant digits:
	\begin{center}
		$S_{b} = \lceil S_d * \log_2{10}\rceil$
	\end{center}

	\bigskip

	From relative error tolerance:
	\begin{center}
		$S_{b} = \lceil \log_2{\left(\frac{100}{\Delta{r}}\right)}\rceil$
	\end{center}

	, where\\

	\bigskip

	\begin{tabular}{ll}
		$S_b$ & Significant bits \\
		$S_d$ & Significant (decimal) digits \\
		$\Delta{r}$ & Relative error tolerance (in percent)
	\end{tabular}

\end{frame}

\end{document}
