\documentclass[compress]{beamer}

\usetheme{Hamburg}

\usepackage[utf8]{inputenc}
\usepackage{units}

\title{Titel des Vortrags}
\author{Name des Autors}
\institute{Arbeitsbereich Wissenschaftliches Rechnen\\Fachbereich Informatik\\Fakultät für Mathematik, Informatik und Naturwissenschaften\\Universität Hamburg}
\date{2012-01-01}

\titlegraphic{\includegraphics[width=0.5\textwidth]{logo}}

\begin{document}

\begin{frame}
	\titlepage
\end{frame}

\begin{frame}
	\frametitle{Gliederung (Agenda)}

	\tableofcontents[hidesubsections]
\end{frame}

\section{Einleitung}
\subsection*{}

\begin{frame}
	\frametitle{Einleitung}

	\begin{itemize}
		\item Hauptpunkt 1

		\begin{itemize}
			\item Unterpunkt 1
			\item Unterpunkt 2
		\end{itemize}
	\end{itemize}

	\begin{figure}
		\begin{center}
			\includegraphics[width=0.75\textwidth]{logo.jpg}
		\end{center}
		\caption{Beispielgrafik}
		\label{fig:logo}
	\end{figure}
\end{frame}

\section{Hauptteil}
\subsection*{}

\begin{frame}[fragile]
	\frametitle{Hauptteil}

	\begin{itemize}
		\item Hauptpunkt 1

		\begin{itemize}
			\item Unterpunkt 1
			\item Unterpunkt 2
		\end{itemize}

		\item Beispielquelltext:
			\begin{verbatim}
			int foo (void)
			{
					return 0;
			}
			\end{verbatim}
	\end{itemize}
\end{frame}

\section{Zusammenfassung}
\subsection*{}

\begin{frame}
	\frametitle{Zusammenfassung}

	\begin{itemize}
		\item Zusammenfassung 1

		\begin{itemize}
			\item Unterpunkt 1
			\item Unterpunkt 2
		\end{itemize}

		\item Zusammenfassung 2

		\begin{itemize}
			\item Unterpunkt 1
			\item Unterpunkt 2
		\end{itemize}
	\end{itemize}
\end{frame}

\section{Literatur}
\subsection*{}

\begin{frame}
	\frametitle{Literatur}

	\nocite*
	\bibliographystyle{alpha}
	\bibliography{literatur}
\end{frame}

\end{document}
