\documentclass[compress]{beamer}

\usetheme{Hamburg}

\usepackage[utf8]{inputenc}
\usepackage{units}

\title{SCIL - Scientific Compression Interface Library}
\author{Armin Schaare}
\institute{Arbeitsbereich Wissenschaftliches Rechnen\\Fachbereich Informatik\\Fakultät für Mathematik, Informatik und Naturwissenschaften\\Universität Hamburg}
\date{2015-11-04}

\titlegraphic{\includegraphics[width=0.5\textwidth]{logo}}

\begin{document}

\begin{frame}
	\titlepage
\end{frame}

\begin{frame}
	\frametitle{Gliederung (Agenda)}

	\tableofcontents[hidesubsections]
\end{frame}

\section{Ziel / Aufgabenstellung}
\subsection*{}

\begin{frame}
	\frametitle{Ziel}

	Implementation einer C-Bibliothek zur Kompression von wissenschaftlichen Daten
\end{frame}

\begin{frame}
	\frametitle{Aufgabenstellung}

	\begin{itemize}[<+->]
		\item Kompression der Daten
		\begin{itemize}[<+->]
			\item Verlustbehaftet
			\item Effizient (Kompressionsrate und -Geschwindigkeit)
		\end{itemize}
		\item Benutzerschnittstelle
		\begin{itemize}[<+->]
			\item Eingabe: Array an float/double Werten
			\item Steuerung: relative/absolute Fehlertoleranz
			\item Ausgabe: komprimiertes Array
		\end{itemize}
	\end{itemize}
\end{frame}

\section{Realisierung}
\subsection*{}

\begin{frame}
	\frametitle{Zeitplan}

	\begin{itemize}[<+->]
		\item<1-5> 02.12.15 Kompression mit absoluter Fehlertoleranz
		\item<2-4> 06.01.16 Kompression mit relativer Fehlertoleranz
		\item<3-4> 24.02.16 Nützliche Subtools
		\item<4> 30.03.16 Feinschliffe und Zeit zur Bewältigung aufgetretener Probleme
	\end{itemize}
\end{frame}

\begin{frame}
	\frametitle{Kompression mit absoluter Fehlertoleranz}

	Prinzip:

	\begin{itemize}[<+->]
		\pause
		\item Finde Minimum und Maximum der Daten
		\item Berechne Anzahl diskreter Werte (mithilfe Fehlertoleranz)
		\item Berechne benötigte Bits pro Zahl
		\item Berechne benötigte Bytes des komprimierten Arrays
		\item Bilde alle Werte auf verlustbehaftete Werte ab und speichere diese bitweise
	\end{itemize}
\end{frame}

\begin{frame}
	\frametitle{Kompression mit absoluter Fehlertoleranz}

	Angenommen 3 Bits zur Darstellung einer Zahl:
	\bigskip\mbox{}
	\pause

	\begin{tabular}{|ccccccccccccccccc}
		\hline
		0 & 0 & \multicolumn{1}{c|}{0} & 0 & 0 & \multicolumn{1}{c|}{1} & 0 & 1 & \multicolumn{1}{c|}{0} & 0 & 1 & \multicolumn{1}{c|}{1} & 1 & 0 & \multicolumn{1}{c|}{0} & 0 & ... \\ \hline
		\hline
		* & * & * & * & * & * & * & \multicolumn{1}{c|}{*} & * & * & * & * & * & * & * & \multicolumn{1}{c|}{*} & ... \\ \hline
	\end{tabular}
	\bigskip\bigskip\mbox{}
	\pause 

	Angenommen 11 ... :
	\bigskip\mbox{}
	\pause

	\begin{tabular}{|ccccccccccccccccc}
		\hline
		0 & 1 & 0 & 0 & 0 & 1 & 1 & 0 & 1 & 1 & \multicolumn{1}{c|}{0} & 0 & 1 & 0 & 0 & 0 & ... \\ \hline
		\hline
		* & * & * & * & * & * & * & \multicolumn{1}{c|}{*} & * & * & * & * & * & * & * & \multicolumn{1}{c|}{*} & ... \\ \hline
	\end{tabular}

\end{frame}

\begin{frame}
	\frametitle{Kompression mit absoluter Fehlertoleranz}

	Erzielte Kompression:
	\pause

	\begin{center}
		\LARGE{$S=\frac{\lceil\log_2{(1 + \frac{max - min}{2 * tol})}\rceil}{U_{bit}}$}\large{ , mit}
	\end{center}

	\begin{tabular}{ll}
		$min$: & Minimum der vorkommenden Daten \\
		$max$: & Maximum der vorkommenden Daten \\
		$tol$: & Absolute Fehlertoleranz \\
		$U_{bit}$: & Anzahl der Bits in floats bzw. doubles \\
		$S$: & Speicherbedarf, relativ zu den unkomprimierten Data
	\end{tabular}

\end{frame}

\begin{frame}
	\frametitle{Kompression mit absoluter Fehlertoleranz}

	\begin{columns}[T]
	\column{.5\textwidth}
		Beispiel:
		\bigskip
		\pause

		Temperaturvariable, mit
		\bigskip\mbox{}
		\begin{tabular}{lcr}
			$min$ & $=$ & $-100$ \\
			$max$ & $=$ & $100$ \\
			$tol$ & $=$ & $0.01$ \\
			$U_{bit}$ & $=$ & $64$
		\end{tabular}
		\pause

	\column{.5\textwidth}
		Es ergibt sich
		\bigskip\mbox{}
		\begin{tabular}{lcl}
			$S$ & $=$ & $\frac{\lceil\log_2{(1 + \frac{max - min}{2 * tol})}\rceil}{U_{bit}}$ \\
			& & \\
			& $=$ & $\frac{\lceil\log_2{10001}\rceil}{64}$ \\
			& & \\
			& $=$ & $\frac{14}{64}$ \\
			& & \\
			& $=$ & $0.21875$
		\end{tabular}
	\end{columns}
	\bigskip
	\pause

	\begin{center}
		\LARGE{Datenkompression um Faktor $\sim$\underline{5}}
	\end{center}

\end{frame}

\section{Fazit}
\subsection*{}

\begin{frame}
	\frametitle{Fazit}

	\begin{itemize}[<+->]
		\item Großes Potential zur Datenkomprimierung
		\item Schnell durch $O(n)$ Zeitaufwandt
	\end{itemize}
\end{frame}

\begin{frame}
	\frametitle{Fragen}

	\begin{center}
		\Huge{Fragen?}
	\end{center}
\end{frame}

\end{document}
