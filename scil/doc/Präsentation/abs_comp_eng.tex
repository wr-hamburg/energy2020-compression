\documentclass[compress]{beamer}

\usetheme{Hamburg}

\usepackage[utf8]{inputenc}
\usepackage{units}

\title{SCIL - Scientific Compression Interface Library}
\author{Armin Schaare}
\institute{Arbeitsbereich Wissenschaftliches Rechnen\\Fachbereich Informatik\\Fakultät für Mathematik, Informatik und Naturwissenschaften\\Universität Hamburg}
\date{2015-11-04}

\titlegraphic{\includegraphics[width=0.5\textwidth]{logo}}

\begin{document}

\begin{frame}
	\titlepage
\end{frame}

\begin{frame}
	\frametitle{Outline}

	\tableofcontents[hidesubsections]
\end{frame}

\section{Goal / Task}
\subsection*{}

\begin{frame}
	\frametitle{Goal}

	Implementation of a C-library for compression of scientific data
\end{frame}

\begin{frame}
	\frametitle{Task}

	\begin{itemize}[<+->]
		\item Compression of the data
		\begin{itemize}[<+->]
			\item Lossy
			\item Efficient (compression rate and speed)
		\end{itemize}
		\item Interface
		\begin{itemize}[<+->]
			\item Input: array of float/double values
			\item Control: relative/absolute error tolerance
			\item Output: compressed array
		\end{itemize}
	\end{itemize}
\end{frame}

\section{Realisation}
\subsection*{}

\begin{frame}
	\frametitle{Timetable}

	\begin{itemize}[<+->]
		\item<1-5> 02.12.15 Compression with absolute error tolerance
		\item<2-4> 06.01.16 Compression with relativer error tolerance
		\item<3-4> 24.02.16 Useful subtools
		\item<4> 30.03.16 Final touch and time for coping of arised problems
	\end{itemize}
\end{frame}

\begin{frame}
	\frametitle{Compression with absolute error tolerance}

	Concept:

	\begin{itemize}[<+->]
		\pause
		\item Find minimum and maximum of the data
		\item Calculate number of discrete values (by means of error tolerance)
		\item Calculate needed bit count per number
		\item Calculate needed Bytes of the whole compressed array
		\item Map all values to lossy compressed ones and store those

	\end{itemize}
\end{frame}

\begin{frame}
	\frametitle{Compression with absolute error tolerance}

	Assume 3 bits to represent a number:
	\bigskip\mbox{}
	\pause

	\begin{tabular}{|ccccccccccccccccc}
		\hline
		0 & 0 & \multicolumn{1}{c|}{0} & 0 & 0 & \multicolumn{1}{c|}{1} & 0 & 1 & \multicolumn{1}{c|}{0} & 0 & 1 & \multicolumn{1}{c|}{1} & 1 & 0 & \multicolumn{1}{c|}{0} & 0 & ... \\ \hline
		\hline
		* & * & * & * & * & * & * & \multicolumn{1}{c|}{*} & * & * & * & * & * & * & * & \multicolumn{1}{c|}{*} & ... \\ \hline
	\end{tabular}
	\bigskip\bigskip\mbox{}
	\pause 

	Assume 11 ... :
	\bigskip\mbox{}
	\pause

	\begin{tabular}{|ccccccccccccccccc}
		\hline
		0 & 1 & 0 & 0 & 0 & 1 & 1 & 0 & 1 & 1 & \multicolumn{1}{c|}{0} & 0 & 1 & 0 & 0 & 0 & ... \\ \hline
		\hline
		* & * & * & * & * & * & * & \multicolumn{1}{c|}{*} & * & * & * & * & * & * & * & \multicolumn{1}{c|}{*} & ... \\ \hline
	\end{tabular}

\end{frame}

\begin{frame}
	\frametitle{Compression with absolute error tolerance}

	Achieved compression:
	\pause

	\begin{center}
		\LARGE{$S=\frac{\lceil\log_2{(1 + \frac{max - min}{2 * tol})}\rceil}{U_{bit}}$}\large{ , with}
	\end{center}

	\begin{tabular}{ll}
		$min$: & Minimum of the data \\
		$max$: & Maximum of the data \\
		$tol$: & Absolute error tolerance \\
		$U_{bit}$: & Number of bits in float resp. double values \\
		$S$: & Memory consumption, relative to the uncompressed data
	\end{tabular}

\end{frame}

\begin{frame}
	\frametitle{Compression with absolute error tolerance}

	\begin{columns}[T]
	\column{.5\textwidth}
		Example
		\bigskip
		\pause

		Temperaturvariable, with
		\bigskip\mbox{}
		\begin{tabular}{lcr}
			$min$ & $=$ & $-100$ \\
			$max$ & $=$ & $100$ \\
			$tol$ & $=$ & $0.01$ \\
			$U_{bit}$ & $=$ & $64$
		\end{tabular}
		\pause

	\column{.5\textwidth}
		Yields
		\bigskip\mbox{}
		\begin{tabular}{lcl}
			$S$ & $=$ & $\frac{\lceil\log_2{(1 + \frac{max - min}{2 * tol})}\rceil}{U_{bit}}$ \\
			& & \\
			& $=$ & $\frac{\lceil\log_2{10001}\rceil}{64}$ \\
			& & \\
			& $=$ & $\frac{14}{64}$ \\
			& & \\
			& $=$ & $0.21875$
		\end{tabular}
	\end{columns}
	\bigskip
	\pause

	\begin{center}
		\LARGE{Data compression of factor $\sim$\underline{5}}
	\end{center}

\end{frame}

\section{Conclusion}
\subsection*{}

\begin{frame}
	\frametitle{Conclusion}

	\begin{itemize}[<+->]
		\item Great potential for data compression
		\item Fast because of $O(n)$ time complexity
	\end{itemize}
\end{frame}

\begin{frame}
	\frametitle{Questions}

	\begin{center}
		\Huge{Questions?}
	\end{center}
\end{frame}

\end{document}
